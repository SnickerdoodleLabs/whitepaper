\section{Introduction}

The current data economy operates an a massive scale and is trending to continue to increase year-over-year \cite{DataAssetValuation}\cite{DataEconomyCitzenConsumer}\cite{DataEconomyValueOfData}. 
In much the same way many corporations are valued more for their financial power and assets than their businesses (airlines, fast food, etc.), 
information businesses have come to be valued primarily for the data they have accumulated. From location history to 
search queries, user data has become a key asset for evaluating Big Tech and information businesses \cite{vectorsOfDigitalTransformation}\cite{industrialPolicyDataDriven}. 
In any economy, value generation is driven by the extraction and utilization of resources. In an information economy, the primary resource is data. 

In the second internet age (”Web2”), the data economy has been largely characterized by consolidation \cite{ConsolitdationWeb}\cite{ConsolitdationInternetEconomy}. 
Much like the post-industrial resource economies of the United States in the early 20th century, the digital information space has been heavily dominated by a small 
number of large corporate entities focused on monetizing business intelligence that can be extracted from large data sets \cite{DataAssetValuation}. Large corporations 
collect, store, and manage immense amounts of data within consolidated, permissioned, data lakes, and use this information advantage to wield immense market power and 
maintain dominance \cite{DataOwnershipBigTechAccessControl}. Unlike physical resources, however, digital data is infinitely copy-able, usable, and transferable. This 
presents a number of challenges within the realms of security, privacy, and overall economic equity.

Despite the rapid increase in collection capabilities and the immense value of actionable information embedded within the data, the role of the individual within the 
traditional data economy has been largely minimized \cite{DataOwnershipBigTechAccessControl}\cite{DataEconomyCitzenConsumer}. While new technologies have led to an explosion 
in the amount of data that can be collected and organized, the originator of said data has limited control and visibility into its collection, its usage, or the distribution of its 
value \cite{auxier_americans_2019}. Users are not equitably rewarded for their role within the data economy, which is demonstrated by the surplus-value of the 
entities who dominate the data markets today \cite{DataOwnershipBigTechAccessControl}\cite{DataEconomyValueOfData}\cite{DataAssetValuation}. In addition, collection, management, 
and storage are all often third-party activities, leading to de-facto ownership through the consolidation of roles.

In addition, the lack of transparency and accountability once the data has been extracted has led to widespread concerns surrounding privacy and security within the 
existing data economy \cite{auxier_americans_2019}\cite{ConsolitdationWeb}. Large companies like Facebook can lose track of where the data of their users is stored 
\cite{FacebookDoesntKnow}. Without strong and auditable security and privacy controls to protect user data, users are at risk and could be exposed to preventable data 
breaches, fraud, identity theft, and malicious data usage. The Synamint Protocol, which will be referred to as "the Protocol", aims to reorient the data economy towards a more user-driven model. By putting 
the user in control of the data they generate, the protocol will allow individuals to properly consent to usage, claim equitable compensation for their participation 
in the data economy, and gain transparency into how their data is leveraged. Additionally, businesses and other data-consuming entities will benefit through implicit 
regulatory compliance and increased competition in and access to the data market.

Recent innovation in the fields of distributed computing and cryptography have led to the emergence of large-scale permissionless, trustless, and highly available 
decentralized networks \cite{SANTANA2022121806}. This has led to utility most notably in the form of new digital asset classes and markets, but has also resulted 
in new forms of digital identity, signature schemes, distributed governance, and auditable computation \cite{Politou2022}. A token economy will serve as the 
compensation mechanism for actors within the user data economy and provide governance functionality, while smart contracts and non-fungible tokens will serve as 
the backbone of auditable and transparent consent mechanisms for data sharing. However, in general, an increase in transparency typically results in a decrease 
in privacy, making it critical that data be stored in a private and secure fashion. Thus, the Protocol will serve as a decentralized data control and coordination 
plane to facilitate the sharing of data, while individual users and the entities they engage with will collectively form the data delivery plane it administers. 

This paper will discuss the proposed protocol that allows individuals to securely and privately collect, manage, and store their data and allow interested parties 
to safely extract value from it. In addition, the protocol will equitably compensate all actors within the system, and seeks to minimize the monopolistic tendencies 
of the existing data economy. The Protocol aims to create a decentralized user data economy that is open and benefits all.