\section{Tokenomics}
\label{section:tokenomics}

% Overview
% What token will do / our objectives
%   Pay for making queries
%       Pay more for more consent (Pay Per View)
%       ? Pay more for more responses (Pay Per Click) ?
%   ? Reward for giving data ? 
%   Helps with DAO
% DAO
%   - Treasury
%   - Governance - 1:1 voting
%   - Control upgrades + on-chain activity
% Distribution
%   - How are we breaking down who gets what of token

% Probably not now but I wonder if we want to include expanding incentives for people who give verified data or store data...  
To ensure equitable compensation and the proper incentivization of all actors, the Snickerdoodle Protocol will rely on its token as a unit of exchange. This "Doodle token" represents access to data and is proportional to the quantity of data. In addition, these tokens will be used by the Snickerdoodle DAO during voting, ensuring that all participants are represented in the protocol's governance. 

\subsection{Utility of The Doodle}
The Doodle token offers utility to participants in the protocol in many ways but primarily acts as a payment system for data insights. Subscribers will pay Doodles proportional to the amount of data they are trying to utilize and the size of their data pool, while data owners can get paid Doodles as a reward for sharing data. Payment for queries can work in multiple ways. We can support a Pay-Per-View model by making subscribers pay Doodles proportional to the number of people that have consented to their data contract. We can support a Pay-Per-Click model by making subscribers pay Doodles proportional to the number of people that respond to the query. In the initial version of the protocol, we will be performing the data acquisition cost via a Pay-Per-View model because it is quicker to implement, and the initial queries will offer less targeting for privacy reasons. % easier to implement and not as many rejected queries

In the future, we may include staking mechanics to improve data safety and increase the incentivize for good behavior. There may also be other Doodle incentive mechanisms used early on to incentivize market adoption.

\subsection{Snickerdoodle DAO}
\label{section:TokenDAO}
Another use of the Doodle token will be as shares in the Snickerdoodle DAO. Each token will represent a single vote in the DAO (MAKE SURE THIS IS TRUE), and the voting scheme may change based on governance proposals. The DAO will also control all aspects of the protocol. At the start, this will be updates to the protocol's contracts, the plugin security model, and the treasury. In the future, this will include valid data schemes, query functions, and valid ingestors. 

\subsection{Distribution}
TODO FIGURE
The breakdown of how the token is being distributed hasn't been ironed out yet. TODO LOOK AT LITE PAPER